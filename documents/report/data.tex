\section{Data}

The base time unit for Kepler data is Barycentric Julian Date (\si{\bjd}), which corrects for the motion of the Kepler satellite around
the solar system's center of mass, and the light curve files contain further offset time. \autocite{keplerManual}

% planet name, period, period percent err, radius, radius percent err
\addvbuffer[12pt]{\begin{tabu} to \linewidth {X[l]*{6}{X[r]}}
\toprule
& \multicolumn{3}{c}{Period} & \multicolumn{3}{c}{Radius} \\
\cmidrule(r){2-4} \cmidrule(r){5-7}
Planet & Calculated [\si{\day}] & Accepted [\si{\day}] & Percent Error [\si{\percent}] & Calculated [\si{\jupiterRadius}] & Accepted [\si{\jupiterRadius}] & Percent Error [\si{\percent}] \\
\midrule
((* for planet in all_planets *))
((( planet.target.name ))) & \num{((( planet.target.periodogram.best_period|sigfigs(4) )))} & \num{((( planet.target.truth.period|sigfigs(4) )))} & \SI{((( planet.period_err|sigfigs(4) )))}{\percent} & \num{((( planet.planet_radius|sigfigs(4) )))} & \num{((( planet.target.truth.radius|sigfigs(4) )))} & \SI{((( planet.radius_err|sigfigs(4) )))}{\percent} \\
((* endfor *))
\bottomrule
\end{tabu}}


((* for planet in all_planets *))
% \subsection{((( planet.target.name )))}

\begin{figure}[H]
    \centering
    \includegraphics[width=0.6\textwidth]{((( planet.light_curve_image )))}
    \caption{This is the cleaned, normalized, and detrended light curve data
    obtained from Kepler. The blue and green dots represent the ingresses and
    egresses of transits.}
\end{figure}

\begin{figure}[H]
    \centering
    \includegraphics[width=0.6\textwidth]{((( planet.periodogram_image )))}
    \caption{The periodogram was generated from the data used in the above light curve. The green dot marks the highest peak, which corresponds to
    the best period detected. Periodograms plot power against frequency, not time, so the x-axis units correspond to inverse days and the values
    are the reciprocal of the period. The best period detected was \SI{((( planet.target.periodogram.best_period|sigfigs(4) )))}{\day}, with a power of
    \SI{((( planet.target.periodogram.best_power|sigfigs )))}{ (\electron\per\second) \tothe{2} \per{\day\tothe{-1}} }.
    Note that power is on a logarithmic scale.}
\end{figure}

((* endfor *))
