\section{Conclusion}

The well-defined peaks on the periodograms show that transits were clearly detected. Kepler-6b in particular shows a single
best peak with a period of \SI{((( kepler6b.target.periodogram.best_period|sigfigs(4) )))}{\day} and a power of
\SI{((( kepler6b.target.periodogram.best_power|sigfigs(4) )))}{ (\electron\per\second) \tothe{2} \per \day\tothe{-1} }. While Kepler 12b
shows three peaks that appear to be of roughly equal height, the highest corresponded to the true period of the planet, since the detected
period of \SI{((( kepler12b.target.periodogram.best_period|sigfigs(4) )))}{\day} is within \SI{1}{\percent} of the actual period, \SI{((( kepler12b.target.truth.period|sigfigs(4) )))}{\day}.
Since the Kepler-12 system has no other known exoplanets, the other peaks may be noise in the data. \autocite{exoplanetArchive}.
In addition, the systems chosen have clearly visible transits on the light curve. Since both Kepler-6b and Kepler-12b have short periods,
the long-cadence light curves from the Kepler mission contain multiple transits. Longer transits and transits on light curves folded around the period
are more U-shaped, with flat bottoms while the planet blocks its parent star, but at the scale of light curves with multiple transits,
the short-duration transits of Kepler-6b and Kepler-12b appear as vertical lines.

It was to be expected that periodic patterns of some kind would be detected. Kepler-6b and Kepler-12b were among the first planets
detected by the Kepler mission, since both have clearly-visible transits. In addition, the Kepler-6 and Kepler-12 systems each contain
only one known exoplanet, so there was less risk of interference in the light curve from another transit. \autocite{exoplanetArchive}. What is
more important is that the system parameters determined corresponded so closely to the accepted values determined by the Kepler mission
and, in most cases, confirmed by the radial-velocity method. The periods were especially close to the actual values, within \SI{1}{\percent} for
each planet. The determined Kepler-6b period of \SI{((( kepler6b.target.periodogram.best_period|sigfigs(4) )))}{\day} was slightly closer, with a percent error of
\SI{((( kepler6b.period_err|sigfigs(4) )))}{\percent} compared to \SI{((( kepler12b.period_err|sigfigs(4) )))}{\percent} for Kepler-12b. The planetary
radii deviated more, by \SI{((( kepler6b.radius_err|sigfigs(4) )))}{\percent} and \SI{((( kepler12b.radius_err|sigfigs(4) )))} for Kepler-6b and Kepler-12b
respectively. This may be due to the greater amount of intermediate calculations to determine the radius; the period was determined directly from the light curve
as part of the periodogram, while the radius required both the transit depth, which itself depended on information derived from the periodogram, and the radius
of the star.

\section{Possible Sources of Error}

Looking at the light curve for Kepler-12b, there is a gap at around \SI{1417}{\bjd}, indicating that the Kepler telescope was not recording during
this time period. Based on the positions of the other transits, it seems plausible that there would have been at least one more transit in that gap.
The missing transit, or transits, could have altered the periodogram results, which is used to calculate both the orbital period and the radius
of the planet. The available data predicted transits just before and after the gap, so a single transit at the center of the gap instead of those
transits could have lengthened the orbital period determined by the periodogram. Since the transit depth was computed from averages, an additional transit
that deviated noticeably from the others could have significantly altered the transit depth, and thus the planetary radius.

All of the calculations were limited to a single light curve per planet. Since the curves are rarely sequential, an attempt to stitch multiple data
sets together would have left gaps in the combined light curve, which could have altered results as previously mentioned. However, individual light
curves may have been too small to capture some phenomena. The Kepler-6b light curve in particular spans only about 10 days, so the effect of anything
with a longer period would be missed. This could include other exoplanets, although there are no other known exoplanets around Kepler-6. \autocite{exoplanetEncyclopedia}.

Kepler light curves contain two sets of flux data. The first, SAP flux, is taken directly from the aperture photometry, while the second is processed
by the pre-search data conditioning (PDC) system. \autocite{keplerManual}. The PDC system makes corrections to the raw flux to remove known
systematic errors that could interfere with transit searches, and is tuned to detect Earth-sized planets. \autocite{pdc}. Since the purpose of this experiment depended on transit detection, the
PDC flux data was used. However, the PDC system can hide certain non-transit patterns, including some on particularly long or short timescales. \autocite{pdc}
Although the PDC system is intended to improve the quality of transit detection, it could partially alter the detected transits.

\section{Further Areas of Investigation}

While the methods used here worked well for simple single-planet systems like Kepler-12 or Kepler-6, they are untested for more complicated systems
like Kepler-32, which has several planets orbiting the star. Therefore, it is worth investigating how well models that work for a single exoplanet
apply to multi-planet systems. Similarly, transits were clearly visible on the selected light curves. In many real-world cases, transits are not as
clearly visible, so one area to investigate would be performance on noisier data. The original intention of this experiment was to compare the results
of the transit and radial velocity methods, but sufficient spectrum data for a radial velocity time series was unavailable. With access to more spectra,
a model based on radial velocity could be compared to this transit model.
