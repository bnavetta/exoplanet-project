\documentclass[12pt,oneside]{memoir}

\OnehalfSpacing

\usepackage{fontspec}
\defaultfontfeatures{Ligatures=TeX}
\setmainfont{Times New Roman}
\usepackage{mathptmx}

\usepackage[margin=1in]{geometry}

\usepackage{booktabs}
\usepackage{tabu}

\usepackage[labelformat=empty]{caption}
\usepackage{float}
\usepackage{graphicx}
\usepackage{verbatimbox}

% \usepackage[backend=biber, style=phys]{biblatex}
% Probably use phys, ieee, maybe find something for turabian
\addbibresource{sources.bib}

\author{Ben Navetta}
\title{Exoplanet Detection - Working Title}

((* include 'common.tex' *))

\renewcommand{\thesection}{}
\renewcommand{\thesubsection}{}
\renewcommand{\thesubsubsection}{}
\renewcommand{\thefigure}{}
\renewcommand{\thetable}{}

\pagestyle{empty}
\pagenumbering{gobble}

\begin{document}

\begin{flushleft}
    \theauthor
    \par
    \thedate
\end{flushleft}
\par
\begin{center}
    \large\textbf{\thetitle}
\end{center}
\par

\begin{abstract}
    Within the past several years, the study of exoplanets has flourished due to advances in technologies and methodologies, as well as increased public awareness.
    While many of the techniques for exoplanet discovery require several layers of indirection, since the traces of distant planets are incredibly faint, some, such as
    the transit method used by the Kepler mission, are relatively straightforward, both in principle and practice. A basic understanding of signal processing, high-school statistics, and around 200 lines of Python code
    proved sufficient to determine exoplanet system parameters to within 7\% error.
\end{abstract}

\section{Purpose}

The purpose of this experiment was to calculate the parameters of exoplanet systems from Kepler data and compare the determined values to the accepted
values for the planets. Standard practices and methods such as linear detrending and the box-fitting least squares algorithm were used to calculate the
period and radius of exoplanets using light curves from the NASA Exoplanet Archive, which were then compared to the accepted values from the Extrasolar
Planets Encyclopedia.\autocite{exoplanetEncyclopedia, exoplanetArchive}

\subsection{Hypothesis}
If Kepler light curves for single-planet systems are analyzed using a straighforward implementation of basic exoplanet dectection methods, then
the calculated parameters should correspond to the accepted values determined by more advanced transit and radial-velocity analysis.

\section{Background}
% There are several tested methods for detecting exoplanets. Of these methods, the transit and Doppler
shift techniques are the most useful. For different reasons, each method tends to detect large planets
in close orbits, the so-called "Hot Jupiters".

The Doppler shift method, also known as the radial velocity method, relies on several layers of indirection
to detect patterns in the velocity of an exoplanet's host star. In a system containing a planet and a star,
both objects orbit around a common center of mass.\autocite{jplMethods} As a result, the radial velocity
of the star displays a sinusoidal pattern over time. Doppler spectroscopy can be used to determine the radial
velocity of a star by measuring the shifts in spectral lines over time. For sufficiently small velocities,
the amount of shift is related to radial velocity by the equation
\( \lambda_{shift} = \lambda_{rest} \frac{v_{radial}}{c} \)
where \(\lambda_{shift}\) is the shift in the location of the absorbtion lines and
\(\lambda_{rest}\) is the rest wavelength of the spectal line.\autocite{dopplerSpectroscopy}

\begin{figure}[H]
	\centering
	\includegraphics[width=0.5\textwidth]{images/spec_sun_arcturus}
	\caption{This image from Michael Richmond demonstrates the Doppler shift in spectra. \autocite{dopplerSpectroscopy}
	The shift is \( \lambda_{shift} = \lambda_{Arcturus} - \lambda{Sun} = \SI{882.55}{\nano\meter} - \SI{882.4}{\nano\meter} =  \SI{0.15}{\nano\meter}\).
	From that shift, \( v_{radial} = \frac{ \SI{0.15}{\nano\meter} }{ \SI{882.4}{\nano\meter} } \times \SI{3e8}{\kilo\meter\per\second} = \SI{50}{\kilo\meter\per\second} \)}
\end{figure}

While somewhat newer than the radial velocity method, the transit method has gained popularity in recent years
with the advent of missions such as Kepler, launched in 2009. The basic idea is to look for the slight drop
in the apparent brightness of a planet's host star as the planet passes between the star and Earth. Given that the
dimming effect is very small and can last for a short period of time, very sensitive instruments are needed,
and noise in the data is often a problem.\autocite{jplMethods} One advantage of this method is that
many stars can be observed at once, increasing the likelihood of detecting a planet, and Kepler
has detected more than 1,000 exoplanet candidates.\autocite{jplMethods}

To detect exoplanets using the transit method, a light curve plotting flux, or changes in intensity of the
light from the star, against time is generated from telescope images. A periodogram, which plots spectral density against frequency,
is then used to determine the period of the exoplanet. On the periodogram, peaks appear at frequencies corresponding
to the period. Once the period has been determined, further analysis of the light curve can determine the ingresses
and egresses of the transit. The transit depth, a measure of the difference in flux during and outside of a transit,
can be related to the radius of the planet:
\begin{align*}
	\Delta F &= \frac{ F_{\text{no transit}} - F_{\text{transit}} }{ F_{\text{no transit}} } \\
	&= (\frac{R_p}{R_*})^2
\end{align*}
This requires that the radius of the star, \(R_*\), is known in advance, but other methods can be used to determine this.
The BLS (Box-fitting Least Squares) algorithm is often used to search for transits, since it detects periodic differences
between two levels, i.e. the normal apparent brightness of the star and the decreased brighness during a transit.\autocite{bls}

For the purposes of this experiment, the transit method was chosen. This was largely due to the greater availability of data from the Kepler project,
but the transit method is also slightly simpler conceptually, which made it more suitable for a relatively straightforward implementation.

((* include 'background.tex' *))

% \section{Materials}

\begin{itemize}
    \item At least one light curve for each planet of interest. Pre-prepared light curves were downloaded from the NASA Exoplanet Archive, but could
        also be generated from telescope imagery. \autocite{exoplanetArchive}
    \item Software to analyze the light curves. The implementation developed for this experiment has been published on \href{https://github.com/roguePanda/exoplanet-project}{GitHub}.
\end{itemize}

\section{Set-Up}

Since the data for this experiment came from the Kepler mission, a brief overview of the Kepler satellite and processing pipeline will be given.
The Kepler mission was designed to survey a fixed region of the galaxy for several years to detect Earth-sized exoplanets using the transit method. \autocite{keplerManual}
The satellite's sole instrument is a photometer consisting of a CCD array, from which pixels are combined into 30-minute long cadence data and
1-minute short cadence data. \autocite{keplerManual} Once downloaded, the data are calibrated and converted into light curves and target pixel files \autocite{keplerManual}.
The data is eventually made available to the public after a few months. Some of the data presented in this paper were obtained from the Multimission
Archive at the Space Telescope Science Institute (MAST). STScI is operated by the Association of Universities for Research in Astronomy, Inc., under NASA contract NAS5-26555.
Support for MAST for non-HST data is provided by the NASA Office of Space Science via grant NAG5-7584 and by other grants and contracts.
% That last bit is required by the MAST people

\begin{figure}[H]
    \centering
    \includegraphics[width=0.6\textwidth]{images/750603main_Ball_Kepler_A8468_275_lg}
    \caption{The Kepler Satellite \autocite{keplerReactionWheelUpdate}}
\end{figure}

\section{Procedure}

\begin{enumerate}
    \item Light curves for Kepler-6b and Kepler-12b were obtained from the \href{http://exoplanetarchive.ipac.caltech.edu/}{NASA Exoplanet Archive}.
    \item The light curves were cleaned by removing all non-finite data points.
    \item The flux in each light curve was normalized by dividing by the median to aid in detrending and periodogram generation.
    \item The light curves were detrended by using least-squares regression to remove global trends. \autocite{untrendy}
    \item A periodogram of each light curve was generated using the box-fitting least squares algorithm. \autocite{bls, pythonBls}
    \item The periodogram and light curve were plotted so that transits and periods could be observed.
    \item The best period was determined from the periodogram by locating the highest peak.
    \item The transit epoch, ingresses, and egresses were determined from the light curve and period.
    \item The transit depth was calculated using the calculated list of transits.
    \item The estimated planetary radius was calculated from the transit depth and the stellar radius as listed in the Extrasolar Planets Encyclopedia. \autocite{exoplanetEncyclopedia}
    \item The percent errors between observed and actual values of the radius and period were calculated.
\end{enumerate}

((* include 'procedure.tex' *))

((* include 'calculations.tex' *))

((* include 'data.tex' *))

((* include 'conclusions.tex' *))

% Hello, ((( name|escape_tex )))!

\nocite{*}
\printbibliography

\end{document}
