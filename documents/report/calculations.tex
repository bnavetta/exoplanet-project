\section{Calculations}

\subsection{Period}

The period is determined by the BLS algorithm, which does not express well as a calculation. The basic idea behind BLS is to fit a step function
of high and low values to subsets of the time series and minimize a function of the weighted data points and the high and low values. \autocite{bls}

\subsection{Epoch}

The epoch here refers to the time of the first transit in the light curve.
One of the values comupted by the BLS implementation used is the bin index of the start of the transit, in1. Dividing this by the number of bins
,nb, (an input to the algorithm) yields the phase of the start of the transit. The epoch is then computed using this phase, the period, and the time
of the first data point in the light curve.

\begin{align*}
    \text{phase1} &= \text{in1} / \text{nb}                                                                                                                                                        &  \text{epoch} &= t_0 + \text{phase1} \times \text{period} \\
    \text{in1} &=  ((( kepler6b.target.periodogram.values.in1|sigfigs )))                                                                                                                          &  \text{period} &= ((( kepler6b.target.periodogram.best_period|sigfigs ))) \\
    \text{nb} &= ((( kepler6b.target.periodogram.values.nb|sigfigs )))                                                                                                                             &  t_0 &= ((( kepler6b.target.light_curve.time[0]|sigfigs ))) \\
    \text{phase1} &= ((( kepler6b.target.periodogram.values.in1|sigfigs ))) / ((( kepler6b.target.periodogram.values.nb|sigfigs )))                                                                &  &= ((( kepler6b.target.periodogram.values.phase1|sigfigs ))) \\
    \text{epoch} &= ((( kepler6b.target.light_curve.time[0]|sigfigs ))) + ((( kepler6b.target.periodogram.values.phase1|sigfigs ))) \times ((( kepler6b.target.periodogram.best_period|sigfigs ))) &  &= \SI{((( kepler6b.target.periodogram.epoch|sigfigs )))}{\bjd}\\
\end{align*}

\subsection{Duration}

The duration of the transit is calculated from the period and \(q\), the fractional transit duration.

\begin{align*}
    \text{duration} &= \text{period} \times q \\
    &= ((( kepler6b.target.periodogram.best_period|sigfigs ))) \times ((( kepler6b.target.periodogram.values.q|sigfigs ))) \\
    &= \SI{((( kepler6b.target.periodogram.duration|sigfigs )))}{\bjd}
\end{align*}

\subsection{Ingress and Egress}

The ingress and egress times of the \textit{n}th transit can be calculated from the epoch, period, and duration. A margin of \SI{0.2}{\bjd} is added on either side.

For the \nth{3} transit,
\begin{align*}
    \text{ingress} &= \text{epoch} + \text{period} \times (n - 1) - 0.2                                                                                                                  &  & \\
    &= ((( kepler6b.target.periodogram.epoch|sigfigs ))) + ((( kepler6b.target.periodogram.best_period|sigfigs ))) \times 2 - 0.2                                                        &  &= \SI{((( kepler6b.target.periodogram.ingresses[2]|sigfigs(5) )))}{\bjd} \\
    \text{egress} &= \text{epoch} + \text{period} \times (n - 1) + \text{duration} + 0.2                                                                                                 &  & \\
    &= ((( kepler6b.target.periodogram.epoch|sigfigs ))) + ((( kepler6b.target.periodogram.best_period|sigfigs ))) \times 2 + ((( kepler6b.target.periodogram.duration|sigfigs ))) + 0.2 &  &= \SI{((( kepler6b.target.periodogram.egresses[2]|sigfigs(5) )))}{\bjd}
\end{align*}

\subsection{Transit Depth}

The transit depth is defined in terms of the flux during and outside of a transit. Here the flux during transit is defined as the average of the
flux values at the bottom of each transit and the flux outside of transit is defined as the average of the flux values not contained in a transit.

\begin{align*}
    \Delta F &= \frac{ F_{\text{no transit}} - F_{\text{transit}} }{ F_{\text{no transit}} } \\
    &= \frac{((( kepler6b.f_no_transit|sigfigs ))) - ((( kepler6b.f_transit|sigfigs )))}{((( kepler6b.f_no_transit|sigfigs )))} \\
    &= ((( kepler6b.transit_depth|sigfigs )))
\end{align*}
% the transit depth is unitless since its flux / flux

\subsection{Planet Radius}

Usually, the radii of stars are given in terms of solar radii, where \(1 R_{\astrosun} = \SI{6.955e8}{\meter}\), and planetary
radii are given in Jupiter radii, where \(1 R_J = \SI{7.1492e7}{\meter}\). However, radii are converted to meters when performing calculations
involving the two.

\begin{align*}
    \Delta F &= (\frac{R_p}{R_*})^2                                                                                       &  R_p &= \sqrt{\Delta F} \times R_* \\
    \Delta F &= ((( kepler6b.transit_depth|sigfigs )))                                                                    &  & \\
    R_* &= \SI{((( kepler6b.target.star.radius|sigfigs )))}{\solarRadius}                                                 &  &= \SI{((( kepler6b.target.star.radius_meters|sigfigs )))}{\meter} \\
    R_p &= \sqrt{((( kepler6b.transit_depth|sigfigs )))} \times \num{((( kepler6b.target.star.radius_meters|sigfigs )))}  &  & \\
    &= \SI{((( kepler6b.planet_radius_meters|sigfigs )))}{\meter}                                                         &  &= \SI{((( kepler6b.planet_radius|sigfigs )))}{\jupiterRadius}
\end{align*}
